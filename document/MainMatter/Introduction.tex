\chapter*{Introducción}\label{chapter:introduction}
\addcontentsline{toc}{chapter}{Introducción}

% \begin{introduction}
%  Hello, world!
% \end{introduction}
% \section{Background}
% Background
% Related Work
% Motivación
% Problema Cientificos
% Objetivos 
% Contribuciones
% -Aumento de volumen de datos y poder computacional desarrolla la inteligencia artificial
% -Las t'ecnicas de aprendizaje automático para extraer conocimineto de grandes volumenes de datos. 

% En la última década, la disponibilidad masiva de datos y el aumento del
% poder computacional de los ordenadores ha impulsado el desarrollo de la
% inteligencia artificial, y específicamente del aprendizaje automático, a tareas que hasta hace algunos años eran dominio absoluto de los seres humanos.

% para expandir la frontera del conocimiento humano, para extraer conocimiento
% a partir del análisis de grandes volúmenes de datos.

% varios los sectores de la industria que están aplicando técnicas de inteligencia artificial de forma intensiva.

% A escala personal, estas tecnologías tienen el potencial de mejorar la calidad de vida humana, por ejemplo, a través de tratamientos médicos más precisos y personalizados, a partir de datos suministrador continuamente por dispositivos electrónicos como teléfonos y relojes inteligentes.

% Por este motivo la demanda de especialistas en esta área ha aumentado continuamente en los últimos años.

% Actual one: 

En los tiempos actuales existe una cantidad enorme de datos y de disponibilidad computacional que han tenido un gran impacto en el desarrollo del campo de inteligencia artificial; con un destacado papel por parte del aprendizaje automático. Ha sustituido varias labores que eran de dominio humano y a su vez ha mejorado la precisión de los resultados obtenidos. Dicho aprendizaje tiene un uso primordial en el descubrimiento de conocimiento a partir del análisis de grandes volúmenes de datos.\vspace{0.25cm}

El aprendizaje autómatico ha tenido una expansión rápida en los diferentes sectores económicos, tanto en áreas productivas como de logística; con el objetivo de aumentar productividad y eficiencia basándose en el análisis de estos algoritmos. \vspace{0.25cm}

Las personas como individuos también han experimentado las ventajas de su aplicación, por ejemplo en el aŕea médica existen algoritmos para diagnosticar y dar tratamientos a enfermedades. Además, en la cotidianidad se tienen dispositivos que están en constante recopilación de información (p.e. teléfonos y relojes inteligentes) que después es usada para predecir preferencias y necesidades de cada individuo, optimizando considerablemente el tiempo empleado por el humano. \vspace{0.25cm}

% --------------------Machine Learning-----------------------------------------

% Optimizing Natural Language Processing
% Pipelines: Opinion Mining Case Study

% Text mining tasks generally require the evaluation of different machine learning and natural language processing algorithms [5]. Selecting the right algorithms for each task requires either a comprehensive knowledge of the problem domain, previous experience in similar contexts, or trial and error [4]. These algorithms interact with each other creating a complex pipeline [11]. As the possible combinations of algorithms for creating a pipeline could vary widely [7], it is unfeasible to evaluate all of them. Evaluating only a few combinations is risky if the com-binations are manually selected because we may miss the best solutions.


% AutoML strategy based on grammatical evolution: A case study about
% knowledge discovery from text

% Besides,
% each algorithm implemented involves parameters
% that need to be manually tuned to find their op-
% timal values, which increases the experimentation
% time considerably. This extensive experimentation
% could be better handled by means of Automated
% Machine Learning (AutoML) strategies.

% General-purpose Hierarchical Optimisation of Machine Learning
% Pipelines with Grammatical Evolution

% The research process for solving a machine learning problem often involves experimenting with several different approaches (neural networks, supervised classifiers, clustering algorithms, etc.). Designing an effective solution to any one of these problems often involves a large exper-imentation phase where researchers use different datasets, algorithms, and specific parameters.
% As an example, Figure 1 shows a hypothetical pipeline composed of several steps. In each step, different options are available. Suitable combinations of these options yield different values for the performance metric that is being evaluated. These hypothetical steps can range from applying some data preprocessing techniques to selecting specific algorithms and further determining the values of their hyperparameters. Exploring all the possible combinations of algorithms and parameters for a given problem can
% be unfeasible, since the number of possibilities is often exponential with respect to the number of steps in a pipeline. Furthermore, when some algorithms have numerical (discrete or continuous) parameters (e.g., regularisation rate, number of neurons in a neural network layer), it is impossible to evaluate all the combinations. This problem is further complicated by the fact that each experiment may have a high computational complexity (e.g., training from scratch a neural network on a large dataset). Researchers are often compelled to select a small number of possibilities based on previous experience and domain knowledge about the problem at hand.

% aprendizaje automático (en inglés machine learning) que consiste
% en el desarrollo de sistemas computacionales capaces de aprender automáti-
% camente a partir de datos [7] 

% la formación de un especialista en este campo requiere de años de entrenamiento en temas que varían desde la teoría matemática subyacente hasta la ingeniería de software necesaria para desplegar sistemas de aprendizaje automático eficientes y escalables en la práctica.

% En la mayoría de los problemas se pueden aplicar múltiples algoritmos de aprendizaje automático, que a su vez tienen parámetros que deben ser configurados por los especialistas para un funcionamiento óptimo.

% La tarea de explorar todas las combinaciones posibles de
% algoritmos y parámetros para un problema dado puede ser impracticable,
% ya que el número de posibilidades es a menudo exponencial con respecto al
% número de pasos en una solución [8]

% Los investigadores en este campo a menudo se
% ven obligados a seleccionar un pequeño número de posibilidades basándose
% en la experiencia previa y el conocimiento del dominio sobre el problema en
% cuestión.

% Actual one:

El aprendizaje automático (en inglés machine learning) consiste
en el desarrollo de sistemas computacionales capaces de aprender automáticamente a partir de datos [7](De la tesis.). Su base está centrada en escoger los algoritmos correctos para cada tarea; que requiere un extenso conocimiento del dominio del problema, experiencia en contextos similares, o el uso de prueba y error [4]. Estos algoritmos interactúan entre ellos para crear un flujo (pipeline) complejo [11](Opinion Mining Case Study). Como las possibles combinaciones de algoritmos para crear dicho flujo puede variar de forma significativa [7](Opinion Mining Case Study), no es práctico evaluarlos todos ya que es muy común es una cantidad exponencial con respecto al número de pasos en una solución[8](de la tesis). Además cada algoritmo implementado requiere la refinación manual de varios parámetros, lo que incrementa el tiempo de experimentación considerablemente. (A case  study about knowledge discovery from text). \vspace{0.25cm}


Todo este proceso es monitoreado e implementado por especialistas del campo que requieren años de entrenamiento en temas que varían desde la teoría matemática subyacente hasta la ingeniería de software necesaria para desplegar sistemas de aprendizaje automático eficientes y escalables en la práctica. (AutoML strategy based on grammatical evolution: A case study about
knowledge discovery from text) \vspace{0.25cm}


El continuo desarrollo de nuevos algoritmos y técnicas de aprendizaje automático , y la variedad de herramientas y datos trae nuevas oportunidades y retos para los investigadores y practicantes tanto en la academia como la industria. Seleccionar la mejor estrategia para resolver un problema se ha tornado crecientemente difícil debido a los ya mencionados problemas referidos al tiempo de implementación y experimentación, sumando además la necesidad de un conocimiento profundo y técnico. (An application to natural language processing) \vspace{0.25cm}








% ----------------------AutoML-------------------------------------------------

% AutoML strategy based on grammatical evolution: A case study about
% knowledge discovery from text

% AutoML is a recent strategy used for the auto-
% matic selection of the best combinations of algo-
% rithmic pipelines. The AutoML process is based
% on the definition of a solution space where all
% the possible pipelines are represented, and a op-
% timization process to explore this space.


% General-purpose Hierarchical Optimisation of Machine Learning
% Pipelines with Grammatical Evolution

% The automation of this lengthy experimentation process is denominated Automatic Machine
% Learning (AutoML). AutoML is an increasingly growing field that has been applied for finding
% optimal machine learning pipelines in a variety of scenarios. As an example, in computer vision,
% where several neural network architectures have been extensively explored, Zoph et al. [44] ap-
% plies reinforced learning to learn to build the optimal neural network architecture for any given
% image dataset. Likewise, general frameworks such as Auto-Sklearn [12], Auto-Weka [39], or
% Auto-Keras [15] have appeared, based upon existing machine learning frameworks, that auto-
% matically explore different combinations of algorithms available in such frameworks.

% Automatic Discovery of Heterogeneous Machine Learning Pipelines:
% An Application to Natural Language Processing

% The continued development of new machine learning algorithms and techniques, and the widely available
% tools and datasets have brought new opportunities and challenges for researchers and practitioners in
% both academia and industry. Selecting the best possible strategy to solve a machine learning problem is
% increasingly difficult partly because it requires long experimentation times and deep technical knowledge.
% In this scenario, Automatic Machine Learning (AutoML) has risen to prominence as it provides tools
% based on specific technologies to efficiently search large spaces of machine learning pipelines, such
% as Auto-Weka (Thornton et al., 2013), Auto-Sklearn (Feurer et al., 2015) and Auto-Keras (Jin et al.,
% 2018). 

% However, practical problems often require combining and comparing heterogeneous algorithms
% implemented with different underlying technologies. Natural language processing is one scenario where
% the space of possible techniques to apply varies widely between different tasks, from preprocessing to
% representation and actual classification. Performing AutoML in an heterogeneous scenario like this is
% complex because the necessary solution could comprise non-compatible tools and libraries. This would
% require all algorithms to agree on a common protocol that enables the output of an algorithm to be shared
% as inputs to any other.




% El AutoML engloba
% el diseño de técnicas para automatizar y facilitar todo el proceso de im-
% plementación, experimentación y despliegue de algoritmos de aprendizaje
% automático, desde la selección de qué algoritmos utilizar, hasta el ajuste de
% sus parámetros internos [9]

% en reducir
% considerablemente el tiempo necesario para la experimentación y el ajuste
% de parámetros a partir de procedimientos de optimización

% acercar a los no-expertos a estas tecnologías al brindar interfaces
% simplificadas a los algoritmos más complejos.

% El objetivo fundamental es
% intercambiar el tiempo del experto por el tiempo de cómputo, automatizando
% todas las tareas de bajo nivel posibles, y dejando al experto las tareas de más
% alto nivel, relacionadas con elegir los enfoques y modelos más apropiados
% para un problema.

% El desarrollo del AutoML en la
% inteligencia artificial promete traer un salto de productividad similar al
% que ocurrió con la llegada de los lenguajes de programación de alto nivel,permitiendo a toda una generación nueva de programadores hacer uso de
% herramientas hasta entonces demasiado complejas.

% Actual one:

El AutoML (automatic machine learning) engloba
el diseño de técnicas para automatizar y facilitar todo el proceso de implementación, experimentación y despliegue de algoritmos de aprendizaje
automático, desde la selección de qué algoritmos utilizar, hasta el ajuste de
sus parámetros internos [9](tésis). Logra acercar a los no-expertos a estas tecnologías al brindar interfaces simplificadas a los algoritmos más complejos; pero su objetivo fundamental es intercambiar el tiempo del experto por el tiempo de cómputo, automatizando
todas las tareas de bajo nivel posibles, y dejando al experto las tareas de más
alto nivel, relacionadas con elegir los enfoques y modelos más apropiados
para un problema. \vspace{0.25cm}


El desarrollo del AutoML en la
inteligencia artificial promete traer un salto de productividad similar al
que ocurrió con la llegada de los lenguajes de programación de alto nivel,permitiendo a toda una generación nueva de programadores hacer uso de
herramientas hasta entonces demasiado complejas. (tesis) AutoML provee instrumentos basados en tecnologías específicas para buscar grandes espacios de flujos de aprendizaje automático con efeciencia, por ejemplo Auto-Weka (Thornton et al., 2013), Auto-Sklearn (Feurer et al., 2015) y Auto-Keras (Jin et al.,2018). Sin embargo, muchos problemas a menudo requieren combinar y comparar algoritmos heterogéneos provenientes de diferentes tecnologías subyacentes. \vspace{0.25cm}







% ---------------AutoML Heterogeneo-----------------------------------

% Automatic Discovery of Heterogeneous Machine Learning Pipelines:
% An Application to Natural Language Processing

% Concretely, we propose AutoGOAL, a system for heterogeneous AutoML in which the user describes
% the input and output of a specific machine problem as well as a performance metric, and the system
% automatically finds the best (or close to best) pipeline of algorithms that solves the problem. This system
% can deal with different machine learning problems by concatenating and composing algorithms from
% several libraries, such as Scikit-learn, NLTK, Keras, and Gensim. It is also flexible, allowing the
% user to introduce new algorithms that seamlessly and automatically integrate within the existing pipelines.
% This is achieved by defining a schema that involves a set of semantic data types and a common protocol.that all algorithms implement, allowing their intercommunication.

% En esta Tesis se propone una extensión al problema clásico de AutoML,
% denominada AutoML Heterogéneo, generalizando los conjuntos de problemas
% que pueden ser resueltos y las estrategias de solución. A continuación se
% define formalmente el problema de AutoML Heterogéneo. Esta definición
% incluye no solo problemas de aprendizaje supervisado sobre datos vectoriales,
% sino escenarios más generales como la recuperación de información.

% Explicar que cosa es y que hace el AutoML Heterogéneo

% Esta definición se aplica directamente a los escenarios de aprendizaje
% supervisado sobre datos vectoriales y tabulares más comunes en tareas
% de AutoML, a la vez que permite modelar problemas donde la entrada
% tenga un tipo semántico de más alto nivel, como texto o imágenes. En el
% aprendizaje supervisado, la entrada consistirá en algunos datos en el espacio
% de características más el objetivo de predicción, permitiendo una fase de
% entrenamiento sobre datos supervisados. Además, esta formulación permite
% modelar problemas donde la salida no es solamente una categoría o valor,
% sino un conjunto de datos semánticos, por ejemplo entidades y relaciones,
% que son evaluados por una métrica a optimizar. De esta manera el AutoML
% Heterogéneo incluye todos los problemas actualmente resueltos por otros
% sistemas de AutoML y amplía el campo de estudio a problemas de aprendizaje
% en nuevos dominios.

% Actual one:

AutoGOAL es un sistema para AutoML heterogéneo donde el usuario describe la entrada y la salida de un problema de máquina específico así como una métrica de rendimiento, y el sistema automáticamente encuentra el mejor (o cerca del mejor) flujo de algoritmos que resuelven el problema. Este sistema puede lidiar
con diferentes problemas de machine learning a través de la concatenación y composición de algoritmos de varias librerías, como Scikit-learn, NLTK, Keras, y Gensim. Además, es flexible; permitiendo al usuario introducir nuevos algoritmos intregándose sin problemas y de manera automática dentro de los flujos existentes. \vspace{0.25cm}


Las formas en las que se pueden construir los diferentes pipelines en dicho sistema se expresan de forma tal que los operadores básicos son operaciones que tienen entrada y salida, que a su vez están definidas sobre tipos. Por lo general, no es suficiente describir dichos tipos con la forma común en los lenaguajes de programación, aquellos que se denominan tipos independientes. De ahí nace la necesidad de utilizar tipos que tengan un predicado asociado. El contenido que traen consigo estos tipos dependientes crea a su vez semánticas distintas y como consecuencia se pueden realizar operaciones distintas. Por ejemplo, un string puede representar un documento, una oración o una secuencia de ADN; y por tanto el mismo tipo de datos puede tener un significado distinto, además de aplicar diferentes operaciones sobre los mismos dadas estas distinciones. \vspace{0.25cm}


\section{Problema Científico:}

El sistema AutoGOAL no dispone de ninguna herramienta que permita expresar de manera flexible el significado de los tipos. Esto a su vez implica que en la construcción de los pipelines se analicen combinaciones que no presentan ninguna incoherencia de tipos, pero que  de cierta manera se puede concluir previamente a su análisis que su resultado no es el esperado. Este problema está dado por la ausencia de una jerarquía de clases que permita a los tipos tener valor semántico, que a su vez se convierten en una optimización al disminuir la cantidad de pipelines válidos, y por tanto decrece el espacio de búsqueda para la solución óptima (o cercana a la óptima).\vspace{0.25cm}


\section{Objetivos:}

El objetivo general de la tesis es el diseño de una jerarquía de clases en AutoGOAL que permita describir tipos con semántica a partir de herramientas de la teoría de tipos dependientes. 

\section{Objetivos específicos:}

- Estudiar el estado del arte de tipos dependientes y los respectivos lenguajes de programación donde se definen dichos tipos. \vspace{0.25cm}


- Identificar del contenido asociado a los tipos dependientes que puede extrapolarse al contexto de AutoGOAL.\vspace{0.25cm}


- Realizar un diseño conceptual del sistema de tipos dependientes que sean compatibles con AutoGOAL.\vspace{0.25cm}


- Realizar una implementación computacional del diseño conceptual dentro del framework de AutoGOAL que sea compatible con el resto de la API.\vspace{0.25cm}


- Realizar un conjunto de experimentos para validar que la implementación no pierde expresividad con respecto a la previamente existente en AutoGOAL y demostrar las nuevas capacidades agregadas obtenidas a través de dicha extensión.\vspace{0.25cm}





%--------------------jerarquia de clases----------------------------